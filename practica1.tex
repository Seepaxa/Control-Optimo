\documentclass{report}
% Configuración de idioma
\usepackage[spanish]{babel}

% Configuración del tamaño de página y márgenes
\usepackage[a4paper,top=2.5cm,bottom=2.5cm,left=3cm,right=3cm,marginparwidth=2cm]{geometry}

% Paquetes útiles
\usepackage{tcolorbox}
\usepackage{listings}
\usepackage{amsmath}
\usepackage{graphicx}
\usepackage[colorlinks=true, allcolors=blue]{hyperref}
\usepackage{fancyhdr}
\usepackage{float}
\usepackage{booktabs}
\usepackage{tikz}
\usetikzlibrary{positioning, calc}
\setlength{\headheight}{1cm} % Ajusta este valor según sea necesario
% Control de espaciado en títulos
\usepackage{titlesec}

% Formato del titulo
\titlespacing*{\chapter}{0pt}{-5pt}{10pt}
\titleformat{\section}{\normalfont\Large\bfseries}{\thesection}{1em}{}
\titleformat{\subsection}[runin]{\normalfont\large\bfseries}{\thesubsection}{1em}{}

%Secciones de titulos
\renewcommand{\thesection}{\arabic{section}}
\renewcommand{\thesubsection}{\arabic{section}.\arabic{subsection}}

% Configuración de encabezado y pie de página
\pagestyle{fancy}
\fancyhf{} % Elimina encabezado y pie predeterminados
\fancyhead[L] {Universidad Católica \\ de Santa Maria }
\fancyhead[C]{Control Optimo \\ y moderno}
\fancyhead[R]{Arequipa-Perú \\ \today}

\fancyfoot[C]{\thepage} % Texto centrado

\begin{document}

% Portada
\begin{titlepage}
    \centering
    \vspace*{2cm}
    % Descripcion de la universidad 
    {\Large Universidad Católica de Santa María \par}
    {\Large Facultad de Ciencias e Ingenierías Físicas y Formales \par}
    {\Large Escuela Profesional de Ingeniería Mecánica, Mecánica-Eléctrica y Mecatrónica \par}

    \begin{figure}[H]
    \centering
    \includegraphics[height=5cm]{escudoucsm.png}
    \end{figure}
    % Título
    {\Huge \bfseries Análisis y Simulación del Modelado en Espacio de Estados para Sistemas Dinámicos \par}
    \vspace{1.5cm}
    

    {\large Curso: Control Moderno y Óptimo \par}
    {\large Docente: Quispe Ccachuco Marcelo Jaime \par}
    {\large Alumno: Chipana Huaylla Sebastián Rodrigo \par}
    {\large Grupo: 02 \par}
    \vfill
    \noindent\rule{\textwidth}{1pt} % Genera una línea de ancho completo y grosor de 1 punto

    % Fecha
    {\large \today \par}
\end{titlepage}

% Tablas de indice y figuras
\tableofcontents % Índice
\listoffigures   % Lista de figuras

  % Inicio de secciones
% Objetivos Didácticos
\section{Objetivos Didácticos}
\noindent\rule{\textwidth}{0.4pt} % Genera una línea más delgada debajo del título de la sección
\begin{itemize}
    \item Realizar el modelo en espacio-estado a partir de las ecuaciones diferenciales.
    \item Implementar un código en MATLAB para simular el sistema ante una entrada.
    \item Realizar la simulación del sistema utilizando Simulink.
\end{itemize}

% Medios Auxiliares
\section{Medios Auxiliares}
\begin{itemize}
    \item PC con MATLAB o acceso a internet.
    \item Libros del curso disponibles en el aula virtual.
\end{itemize}

% Condiciones Generales
\section{Condiciones Generales}
\begin{itemize}
    \item Se han planteado 16 sistemas que serán sorteados para que cada alumno realice el modelamiento en espacio de estados.
    \item La entrega del informe con los anexos se realizará a través del aula virtual del curso.
    \item La práctica tiene una duración de 1 sesión.
\end{itemize}

% Modelado en Espacio-Estado
\section{Modelado en Espacio-Estado}
    \subsection{Ejemplo de Modelamiento en Espacio-Estado}

\begin{figure}[H]
    \centering
    \includegraphics[width=\textwidth]{1 A.png}
    \caption{Diagrama del sistema modelado en espacio-estado.}
    \label{fig:diagrama_modelo}
\end{figure}

Damos un pequeño desplazamiento hacia arriba

\[
a = L_1 \sin(\theta) \approx L_1 \theta
\]
\[
b = L_2 \sin(\theta) \approx L_2 \theta
\]

D.C.L. de la masa \(m\):

\[
K_2(b - y) = m_2 \ddot{y}
\]

2 da ley de newton en la barra:

\[
uL_2 - K_2(b - y)L_2 - K_1 aL_1 = I\ddot{\theta}
\]

Reemplazando \(a\) y \(b\) en las ecuaciones:

\[
K_2(L_2\theta - y) = m_2\ddot{y}
\]

\[
uL_2 - K_2(L_2\theta - y)L_2 - K_1(L_1\theta)L_1 = I\ddot{\theta}
\]

\begin{table}[H]
    \centering
    \begin{tabular}{|c|c|c|}
        \hline
        \textbf{Estados} & \textbf{Salidas} & \textbf{Entradas} \\
        \hline
        \(x_1 = y\) & \(y = y\) & \(u = u\) \\
        \(x_2 = \dot{y} = \dot{x}_1\) & & \\
        \(x_3 = \theta\) & & \\
        \(x_4 = \dot{\theta} = \dot{x}_3\) & & \\
        \hline
    \end{tabular}
    \caption{Relación de Estados, Salidas y Entradas del Sistema}
    \label{tabla:estados_salidas_entradas}
\end{table}

Haciendo un cambio de variables
\[
K_2 (L_2 x_3 - x_1) = m_2 \dot{x}_2
\]
\[
u L_2 - K_2 (L_2 x_3 - x_1) L_2 - K_1 (L_1 x_3) L_1 = I \dot{x}_4
\]

Pasando a espacios de estado

\[
\begin{bmatrix}
\dot{x}_1 \\
\dot{x}_2 \\
\dot{x}_3 \\
\dot{x}_4
\end{bmatrix}
=
\begin{bmatrix}
0 & 1 & 0 & 0 \\
-\frac{K_2}{m_2} & 0 & \frac{K_2 L_2}{m_2} & 0 \\
0 & 0 & 0 & 1 \\
\frac{K_2 L_2}{I} & 0 & -\frac{K_1 L_1^2 + K_2 L_2^2}{I} & 0
\end{bmatrix}
\begin{bmatrix}
x
\end{bmatrix}
+
\begin{bmatrix}
0 \\
0 \\
0 \\
\frac{L_2}{I}
\end{bmatrix} u
\]

\[
y = \begin{bmatrix}
1 & 0 & 0 & 0
\end{bmatrix} \mathbf{x} + \begin{bmatrix}
0
\end{bmatrix} u
\]
Dar los valores para las constantes, y considerar condiciones iniciales.
\section{Realizar simulaciones en MATLAB }
\subsection{Codigo en MATLAB }
\begin{verbatim}
% Declarar datos
B1=0.01; B2=0.2; B3=0.5; 
K1=1; K2=2;
M1=12; M2=8;

% Declarar matrices
A=[0 1 0 0;
 -(K1/M1) -((B1+B2+B3)/M1) 0 (B2/M1);
 0 0 0 1;
 0 (B2/M2) -(K2/M2) -(B2/M2)];

B=[0 0 0 1/M2];
C=[1 0 0 0];
D=0;

% Condición inicial
X0=[0 pi/4 0 pi/4];

% Crear el sistema en espacio de estados
SYS1=ss(A,B,C,D);

% Tiempo y entrada
t=0:0.1:5;
u=2*ones(1,length(t));

% Simulación
step(SYS1);
\end{verbatim}
Ejecutandolo en Matlab
\begin{figure}[H]
    \centering
    \includegraphics[width=\textwidth]{1 B.png}
    \caption{Respuesta al Step de un sistema en espacio de estados.}
    \label{fig:diagrama_modelo}
\end{figure}

\begin{figure}[H]
    \centering
    \includegraphics[width=\textwidth]{1 C.png}
    \caption{Simulación de sistema de espacio de estados en Simulink.}
    \label{fig:diagrama_modelo}
\end{figure}
A continuación, se muestran diferentes sistemas. Se pide que cada alumno escoja un sistema y realice las siguientes actividades:

\begin{enumerate}
    \item Obtención de las ecuaciones diferenciales para cada sistema.
    \item Realizar el modelamiento en espacio de estados a partir de las ecuaciones anteriores.
    \item Dar valores a las constantes y realizar un código en MATLAB para simular el sistema ante una entrada adecuada (escalón, impulso, senoidal, etc.), considerando las condiciones iniciales.
    \item Realizar el ítem anterior utilizando Simulink.
    \item Los problemas se encuentran en el libro \textit{Close and Dean, Parte 1: Modelado y Análisis Parte I}.
\end{enumerate}

\section{Desarrollo del ejercicio}

Ejemplo 2.7 la entrada es la fuerza como un escalón y las salidas son las posiciones.

\begin{figure}[H]
    \centering
    \includegraphics[width=\textwidth]{1 D.png}
    \caption{Ejemplo 2.7}
    \label{fig:diagrama_modelo}
\end{figure}
Son 2 masas 
\begin{figure}[H]
    \centering
    \includegraphics[width=\textwidth]{1 E.png}
    \caption{Fuerzas en la masa 1}
    \label{fig:diagrama_modelo}
\end{figure}
\begin{figure}[H]
    \centering
    \includegraphics[width=\textwidth]{1 F.png}
    \caption{Fuerzas en la masa 2}
    \label{fig:diagrama_modelo}
\end{figure}
Obtendremos 2 masas con una entrada escalon y 2 salidas como las posiciones de ambas masas.

\[
M_1 \dot{v}_1 + B_1 v_1 + k_1 (x_1 - x_2) = f_a(t)
\]
Esta ecuación diferencial representa el equilibrio dinámico de un sistema físico sujeto a fuerzas internas y externas. Desglosando cada componente, obtenemos lo siguiente:
\[
M_2 \dot{v}_2 + B_2 v_2 + k_2 x_2 = k_1 (x_1 - x_2)
\]

\section{Representación en Espacios de Estado}

Consideremos las siguientes ecuaciones diferenciales que describen un sistema dinámico:

\begin{equation}
M_1 \dot{v}_1 + B_1 v_1 + k_1 (x_1 - x_2) = f_a(t)
\label{eq:eq1}
\end{equation}

\begin{equation}
M_2 \dot{v}_2 + B_2 v_2 + k_2 x_2 = k_1 (x_1 - x_2)
\label{eq:eq2}
\end{equation}

Nuestro objetivo es expresar estas ecuaciones en forma de espacios de estado, es decir, en la forma:

\[
\dot{\mathbf{x}} = A\mathbf{x} + B\mathbf{u}
\]

donde \(\mathbf{x}\) es el vector de estado, \(A\) es la matriz del sistema, \(B\) es la matriz de entrada, y \(\mathbf{u}\) es el vector de entrada.

\subsection{Definición de las Variables de Estado}

Primero, definimos las variables de estado como sigue:

\[
x_1 = x_1, \quad x_2 = \dot{x}_1 = v_1, \quad x_3 = x_2, \quad x_4 = \dot{x}_2 = v_2
\]

Aquí, \(x_2\) representa la velocidad \(v_1\) y \(x_4\) representa la velocidad \(v_2\). En términos de estas variables, las ecuaciones \ref{eq:eq1} y \ref{eq:eq2} se pueden reescribir.

\subsection{Reescribiendo las Ecuaciones}

La primera ecuación \ref{eq:eq1} se puede reescribir como:

\[
M_1 \dot{v}_1 = f_a(t) - B_1 v_1 - k_1 (x_1 - x_2)
\]

Sustituyendo \(v_1 = x_2\) y \(\dot{v}_1 = \dot{x}_2\):

\[
M_1 \dot{x}_2 = f_a(t) - B_1 x_2 - k_1 (x_1 - x_3)
\]

\[
\dot{x}_2 = \frac{-B_1 x_2 - k_1 (x_1 - x_3) + f_a(t)}{M_1}
\]

La segunda ecuación \ref{eq:eq2} se reescribe de manera similar:

\[
M_2 \dot{v}_2 = k_1 (x_1 - x_2) - B_2 v_2 - k_2 x_2
\]

Sustituyendo \(v_2 = x_4\) y \(\dot{v}_2 = \dot{x}_4\):

\[
M_2 \dot{x}_4 = k_1 (x_1 - x_3) - B_2 x_4 - k_2 x_3
\]

\[
\dot{x}_4 = \frac{k_1 (x_1 - x_3) - B_2 x_4 - k_2 x_3}{M_2}
\]

\subsection{Construcción del Modelo de Espacios de Estado}

Ahora podemos escribir el sistema completo en forma de espacios de estado. Las ecuaciones de estado son:

\[
\begin{aligned}
\dot{x}_1 &= x_2, \\
\dot{x}_2 &= \frac{-B_1 x_2 - k_1 (x_1 - x_3) + f_a(t)}{M_1}, \\
\dot{x}_3 &= x_4, \\
\dot{x}_4 &= \frac{-B_2 x_4 - k_2 x_3 + k_1 (x_1 - x_3)}{M_2}.
\end{aligned}
\]

Esto se puede expresar en forma matricial como:

\[
\dot{\mathbf{x}} = A\mathbf{x} + B\mathbf{u}
\]

Donde \(\mathbf{x} = \begin{bmatrix} x_1 \\ x_2 \\ x_3 \\ x_4 \end{bmatrix}\) y \(\mathbf{u} = f_a(t)\). Las matrices \(A\) y \(B\) son:

\[
A = \begin{bmatrix}
0 & 1 & 0 & 0 \\
-\frac{k_1}{M_1} & -\frac{B_1}{M_1} & \frac{k_1}{M_1} & 0 \\
0 & 0 & 0 & 1 \\
\frac{k_1}{M_2} & 0 & -\frac{k_1 + k_2}{M_2} & -\frac{B_2}{M_2}
\end{bmatrix}, \quad B = \begin{bmatrix}
0 \\
\frac{1}{M_1} \\
0 \\
0
\end{bmatrix}
\]

La salida del sistema está dada por las posiciones \(x_1\) y \(x_3\), lo que podemos escribir como:

\[
\mathbf{y} = \begin{bmatrix} y_1 \\ y_2 \end{bmatrix} = \begin{bmatrix} x_1 \\ x_3 \end{bmatrix}
\]

\subsection{Definición de la Salida}

La salida del sistema está dada por las posiciones \(x_1\) y \(x_3\), lo que podemos escribir como:

\[
\mathbf{y} = \begin{bmatrix} y_1 \\ y_2 \end{bmatrix} = \begin{bmatrix} x_1 \\ x_3 \end{bmatrix}
\]

\subsection{Matriz de Salida \(C\) y Matriz de Transmisión Directa \(D\)}

En la forma estándar de espacios de estado, la ecuación de salida se expresa como:

\[
\mathbf{y} = C\mathbf{x} + D\mathbf{u}
\]

Dado que las salidas dependen directamente de los estados \(x_1\) y \(x_3\), la matriz \(C\) y \(D\) se definen como:

\[
C = \begin{bmatrix} 1 & 0 & 0 & 0 \\ 0 & 0 & 1 & 0 \end{bmatrix}, \quad D = \begin{bmatrix} 0 \\ 0 \end{bmatrix}
\]


Por lo tanto, la representación completa del sistema en espacios de estado, incluyendo la ecuación de salida, es:

\[
\dot{\mathbf{x}} = A\mathbf{x} + B\mathbf{u}
\]

\[
\mathbf{y} = C\mathbf{x} + D\mathbf{u}
\]

\subsection{Código en matlab}
\begin{verbatim}
% Chipana Huaylla Sebastián Rodrigo - Practica 1 Control Optimo
% Obtencion de salida de matriz de estados

% Definir los parámetros del sistema
B1 = 0.01; % Coeficiente de amortiguamiento B1
B2 = 0.2;  % Coeficiente de amortiguamiento B2
K1 = 1;    % Constante de resorte K1
K2 = 2;    % Constante de resorte K2
M1 = 12;   % Masa M1
M2 = 8;    % Masa M2

% Definir las matrices A, B, C, y D
A = [0 1 0 0;
     -K1/M1 -B1/M1 K1/M1 0;
     0 0 0 1;
     K1/M2 0 -(K1+K2)/M2 -B2/M2];

B = [0; 1/M1; 0; 0];

C = [1 0 0 0; 0 0 1 0]; % Salidas son x1 y x3
D = [0; 0];

% Crear el sistema en espacio de estados
SYS = ss(A, B, C, D);

% Definir el tiempo de simulación y la entrada (fuerza aplicada)
t = 0:0.1:80; % Tiempo de simulación de 0 a 10 segundos con incrementos de 0.1 s
u = 2*ones(size(t)); % Entrada escalón de amplitud 2

% Simular la respuesta del sistema
[Y, T, X] = lsim(SYS, u, t);

% Graficar la respuesta de las posiciones x1 y x3
figure;
subplot(2,1,1);
plot(T, Y(:,1), 'r', 'LineWidth', 1.5); % Respuesta x1
title('Respuesta de la Posición x_1');
xlabel('Tiempo (s)');
ylabel('Posición x_1 (m)');
grid on;

subplot(2,1,2);
plot(T, Y(:,2), 'b', 'LineWidth', 1.5); % Respuesta x3
title('Respuesta de la Posición x_3');
xlabel('Tiempo (s)');
ylabel('Posición x_3 (m)');
grid on;

\end{verbatim}

\subsection{Graficas de posición x1  x3}

\begin{figure}[H]
    \centering
    \includegraphics[width=\textwidth]{1 G.png}
    \caption{Movimiento de la masa 1}
    \label{fig:diagrama_modelo}
\end{figure}

\begin{figure}[H]
    \centering
    \includegraphics[width=\textwidth]{1 H.png}
    \caption{Movimiento de la masa 2}
    \label{fig:diagrama_modelo}
\end{figure}

\section{Conclusión}

La representación en espacios de estado para las ecuaciones diferenciales dadas está ahora completamente definida. Esta forma es útil para el análisis y diseño de controladores, simulaciones y para la comprensión del comportamiento dinámico del sistema.

% Bibliografía
\bibliographystyle{apalike}
\bibliography{bibliografia}

\section{Conclusiones}
\begin{itemize}
    \item Se desarrolló el modelo en espacio de estados para un sistema de masa-resorte-amortiguador, logrando representar las ecuaciones diferenciales de segundo orden en un formato matricial.
    \item El análisis del sistema en espacio de estados permitió estudiar las dinámicas internas y la respuesta del sistema ante diferentes entradas.
    \item Las simulaciones realizadas en MATLAB y Simulink confirmaron la precisión del modelo en espacio de estados, demostrando que el enfoque es adecuado para predecir el comportamiento del sistema bajo diversas condiciones.
    \item Las herramientas computacionales fueron esenciales para visualizar y comprender la evolución temporal de las variables de estado y las salidas del sistema.
    \item La práctica reforzó la comprensión teórica del modelado en espacio de estados y su aplicación en el análisis de sistemas dinámicos.
\end{itemize}


\end{document}